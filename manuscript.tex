\documentclass[]{aiaa-tc}% insert '[draft]' option to show overfull boxes

\usepackage{aiaa_packages}

 \title{Low-Thrust Trajectory Design Near Asteroids Using Invariant Manifolds and Reachability Sets}

 \author{
  Shankar Kulumani\thanksibid{1}%
    \thanks{Doctoral Student, \href{mailto:skulumani@gwu.edu}{skulumani@gwu.edu}. Student AIAA Member.}
  \ and Taeyoung Lee\thanksibid{2}\thanks{Associate Professor, \href{mailto:tylee@gwu.edu}{tylee@gwu.edu}. AIAA Member}\\
  {\normalsize\itshape
   Mechanical and Aerospace Engineering, George Washington University, 800 22nd St NW, Washington DC }\\
   }

 % Data used by 'handcarry' option if invoked
 \AIAApapernumber{YEAR-NUMBER}
 \AIAAconference{Conference Name, Date, and Location}
 \AIAAcopyright{\AIAAcopyrightD{YEAR}}

\begin{document}

\maketitle

\begin{abstract}
Abstract
\end{abstract}

\section*{Nomenclature}

\begin{tabbing}
  XXX \= \kill% this line sets tab stop and adds a newline
  $J$ \> Jacobian Matrix \\
  $f$ \> Residual value vector \\
  $x$ \> Variable value vector \\
  $F$ \> Force, \si{\newton} \\
  $m$ \> Mass, \si{\kilo\gram} \\
  $\Delta x$ \> Variable displacement vector \\
  $\alpha$ \> Acceleration, \si{\meter\per\second\squared} \\[5pt]
  \textit{Subscript}\\
  $i$ \> Variable number \\
\end{tabbing}

\section{Introduction}

% Motivation for missions/studying asteroids
Small solar system bodies, such as asteroids and comets, are of significant interest to the scientific community.
These small bodies offer great insight into the early formation of the solar system.
This insight offers additional detail into the formation of the Earth and also the probable formation of other planetary systems.
Of particular interest are those near-Earth asteroids (NEA) which inhabit heliocentric orbits in vicinity of the Earth.
These easily accessible bodies provide attractive targets to support space industrialization, mining operations, and scientific missions..
NEAs potentially contain many materials useful for propulsion, construction, semiconductors, precious and strategic metals~\cite{ross2001}.
In addition, these NEAs are also of concern for their potential to impact the Earth~\cite{wie2008}.
Asteroids and comets are the greatest threat to future civilizations and as a result there is a focused effort to mitigate these risks.
Operating in the vicinity of asteroids is a challenging problem for spacecraft missions.

% Difficulty in system model 
While there has been significant study of interplanetary transfer trajectories, relatively less analysis has been conducted on operations in the vicinity of asteroids.
The dynamic environment of around asteroids is strongly perturbed and challenging for analysis.
A combination of factors contribute to make operations in the vicinity of small bodies challenging, such as distended body shapes, unknown spin states, and non-gravitational effects such as solar radiation pressure~\cite{scheeres2012}.

% most use a spherical harmonic model but we use a polyhedron model

% Investigate low thrust propulstion for manuevers
% vesta paper
% gravity tractor papers
 
%Reachability set is key to the design of transfers	

	
\section{System Model}

\subsection{Polyhedron Gravity Model}

Cite the original werner papers

\subsection{Asteroid Dynamics}
Cite scheeres and his book

Equations of motion

Jacobi constant

Relate to three body problem

Equilibrium points. Cite work done with triaxial ellipsoids and other Castalia papers

Periodic orbits

\section{Reachability Set}

Show \Poincare section

Optimal control problem 

Discuss the constraints and angles on the four dimensional space of the section

Show partial derivatives 

\section{Numerical Simulation}

Transfer about asteroid Castalia - cite some Scheeres papers

\section{Conclusion}



\section*{Appendix}


\section*{Acknowledgments}

A place to recognize others.

\bibliographystyle{aiaa} 
\bibliography{library}

\end{document}
